%%%%%%%%%%%%%%%%%%%%%%%%%%%%%%%%%%%%%%%%%
% NIH Grant Proposal for the Specific Aims and Research Plan Sections
% LaTeX Template
% Version 1.0 (21/10/13)
%
% This template has been downloaded from:
% http://www.LaTeXTemplates.com
%
% Original author:
% Erick Tatro (erickttr@gmail.com) with modifications by:
% Vel (vel@latextemplates.com)
%
% Adapted from:
% J. Hrabe (http://www.magalien.com/public/nih_grants_in_latex.html)
%
% License:
% CC BY-NC-SA 3.0 (http://creativecommons.org/licenses/by-nc-sa/3.0/)
%
%%%%%%%%%%%%%%%%%%%%%%%%%%%%%%%%%%%%%%%%%

%----------------------------------------------------------------------------------------
%	PACKAGES AND OTHER DOCUMENT CONFIGURATIONS
%----------------------------------------------------------------------------------------

\documentclass[11pt,notitlepage]{article}

% Variabili per non ripetere i contatti mille volte, MODIFICARE QUI
\newcommand{\nomeStudente}{Giovanni Bruno}
\newcommand{\cognomeStudente}{Sanna}
\newcommand{\matricolaStudente}{1029744}
\newcommand{\emailStudente}{giovannibruno.sanna@studenti.unipd.com}
\newcommand{\telStudente}{+39 0000000000}

\newcommand{\nomeTutorAziendale}{Nome}
\newcommand{\cognomeTutorAziendale}{Cognome}
\newcommand{\emailTutorAziendale}{xxtutor@azienda.it}
\newcommand{\telTutorAziendale}{+39 0000000000}

\newcommand{\ragioneSocAzienda}{Soluzioni Software S.R.L.}
\newcommand{\indirizzoAzienda}{Via Via dei Ronchi, 21, 35127 Padova (PD)}
\newcommand{\sitoAzienda}{http://www.soluzioni-sw.it/}



% A note on fonts: As of 2013, NIH allows Georgia, Arial, Helvetica, and Palatino Linotype. LaTeX doesn't have Georgia or Arial built in; you can try to come up with your own solution if you wish to use those fonts. Here, Palatino & Helvetica are available, leave the font you want to use uncommented while commenting out the other one.
%\usepackage{palatino} % Palatino font
\usepackage[utf8x]{inputenc}%codifica
\usepackage{helvet} % Helvetica font
\usepackage{multirow}
\renewcommand*\familydefault{\sfdefault} % Use the sans serif version of the font
\usepackage[T1]{fontenc}
\linespread{1.2} % A little extra line spread is better for the Palatino font
\usepackage{fancyhdr} %pacchetto per le intestazioni
\usepackage{hyperref}
\usepackage{lipsum} % Used for inserting dummy 'Lorem ipsum' text into the template
\usepackage{amsfonts, amsmath, amsthm, amssymb} % For math fonts, symbols and environments
\usepackage{graphicx} % Required for including images
%\usepackage{booktabs} % Top and bottom rules for table
%\usepackage{wrapfig} % Allows in-line images
%\usepackage[labelfont=bf]{caption} % Make figure numbering in captions bold
\usepackage[top=0.5in,bottom=0.5in,left=0.5in,right=0.5in]{geometry} % Reduce the size of the margin
\pagestyle{empty}

\hyphenation{ionto-pho-re-tic iso-tro-pic fortran} % Specifies custom hyphenation points for words or words that shouldn't be hyphenated at all

\hypersetup{
	colorlinks=true,
	linkcolor=black,
	urlcolor=blue
}

%----------------------------------------------------------------------------------------
%	Creato da Mich - Updated by Simone Pessotto 04/08/2015
%----------------------------------------------------------------------------------------

\begin{document}
	
\noindent
\parbox{0.7\columnwidth}{Università degli Studi di Padova\\
	Piano di lavoro stage presso \ragioneSocAzienda{}\\
	\nomeStudente{} \cognomeStudente{} (\matricolaStudente{})}%matricola
\parbox{0.3\columnwidth}{
	\hfill \includegraphics[scale=0.08]{immagini/logo-unipd.png}}

\bigskip
\begin{center}
{\Huge \textbf{Analisi delle piattaforme di messaggistica}} \\ \bigskip
	{\Large \textit{presso \ragioneSocAzienda{}}}\\ \bigskip
	{\Large \textit{\nomeStudente{} \cognomeStudente{}}}
\end{center}

%\section*{Contatti}
%\textbf{Studente:} \nomeStudente{} \cognomeStudente{}, \href{mailto:\emailStudente{}}{\emailStudente{}}, \telStudente{} \\
%\textbf{Tutor aziendale:} \nomeTutorAziendale{} \cognomeTutorAziendale{}, \href{mailto:\emailTutorAziendale{}}{\emailTutorAziendale{}}, \telTutorAziendale{} \\
%\textbf{Azienda:} \ragioneSocAzienda{}, \indirizzoAzienda{}, \href{\sitoAzienda{}}{\sitoAzienda{}}

\section*{Contenuto}
Tale documento contiene una breve relazione sulle piattaforme di messaggistica analizzate:
\begin{itemize}
		\item Telegram Messenger; 
		\item Skype Messenger; 
		\item WeChat;
		\item Facebook Messenger. 
	\end{itemize} 
	Per ogni piattaforma analizzata viene riportata una breve descrizione, vantaggi e svantaggi, in termini di caratteristiche e funzionalità supportate. \\
	Inoltre viene individuata la piattaforma, tra quelle analizzate, scelta per lo sviluppo del bot.


\newpage



\noindent
\parbox{0.7\columnwidth}{Università degli Studi di Padova\\
	Piano di lavoro stage presso \ragioneSocAzienda{}\\
	\nomeStudente{} \cognomeStudente{} (\matricolaStudente{})}%matricola
\parbox{0.3\columnwidth}{
	\hfill \includegraphics[scale=0.08]{immagini/logo-unipd.png}}

\bigskip
\section*{Telegram Messenger}
\subsection*{Descrizione}
Telegram è un servizio di messaggistica istantanea basato su cloud ed erogato senza fini di lucro. I client ufficiali di Telegram sono open-source per diverse piattaforme. Caratteristiche di Telegram sono la possibilità di stabilire conversazioni cifrate ent-to-end, effettuare chiamate vocali cifrate end-to-end, scambiare messaggi vocali, videomessaggi, fotografie, video, stickers e file di qualsiasi tipo grandi fino a 1,5 GB.  \\
Oltre a inviare diversi tipi di file, Telegram è focalizzato sulla velocità. I messaggi si sincronizzano perfettamente in tutti i canali (e a tutti gli utenti) simultaneamente. \\
L'accessibilità è una caratteristica chiave di Telegram. Il suo sistema basato su cloud consente di accedere ai messaggi da qualsiasi dispositivo e da qualsiasi posizione, indipendentemente da dove siano stati originati. \\
La piattaforma è relativamente piccola, con circa 100 milioni di utenti, e non è dominante in molti paesi del mondo. Tuttavia, gli utenti che ha sono ben distribuiti e la sua base di utenti è in crescita.

\subsection*{Vantaggi}
\begin{itemize}
		\item Autorizzazione utente tramite chiave cifrata e numero di telefono;
		\item Supporta una crittografia a doppio strato. La crittografia server-client viene utilizzata nelle chat cloud, mentre le chat segrete utilizzano un ulteriore livello client-client. 
		\item Inline Bot: permette all'utente di inviare query al bot da una qualsiasi conversazione, senza dover sottoscrivere il bot e senza inviare nessun messaggio testuale;
		\item Possibilità di scegliere il data center più vicino a cui connettersi;
		\item Ottimizzazioni client-side: conferma di ricezione di un messaggio semplificata, ottimizzazione del download e upload da/sul server, raggruppamento degli update in assenza di connessione;
		\item Permette di scegliere la modalità di gestione degli update, se webhook o long-polling.
\end{itemize} 

\subsection*{Svantaggi}
\begin{itemize}
		\item Bacino contenuto di utenti;
		\item Documentazione ufficiale povera di contenuti e scarsamente approfondita; 
		\item Non integra NLP (Natural Language Processing) nativamente;
		\item Non fornisce nessun strumento integrato di analisi delle prestazioni del bot e delle interazioni con esso;
		\item Scarsa personalizzazione della UI nella conversazione.
\end{itemize} 

\bigskip
\section*{Skype Messenger}
\subsection*{Descrizione}

\subsection*{Vantaggi}
\begin{itemize}
		\item 
		\item  
		\item 
\end{itemize} 

\subsection*{Svantaggi}
\begin{itemize}
		\item 
		\item  
		\item 
\end{itemize} 

\bigskip
\section*{WeChat}
\subsection*{Descrizione}

\subsection*{Vantaggi}
\begin{itemize}
		\item 
		\item  
		\item 
\end{itemize} 

\subsection*{Svantaggi}
\begin{itemize}
		\item 
		\item  
		\item 
\end{itemize} 

\bigskip
\section*{Facebook Messenger}
\subsection*{Descrizione}

\subsection*{Vantaggi}
\begin{itemize}
		\item 
		\item  
		\item 
\end{itemize} 

\subsection*{Svantaggi}
\begin{itemize}
		\item 
		\item  
		\item 
\end{itemize} 

\bigskip
\section*{Scelta conclusiva}
\end{document}