%%%%%%%%%%%%%%%%%%%%%%%%%%%%%%%%%%%%%%%%%
% NIH Grant Proposal for the Specific Aims and Research Plan Sections
% LaTeX Template
% Version 1.0 (21/10/13)
%
% This template has been downloaded from:
% http://www.LaTeXTemplates.com
%
% Original author:
% Erick Tatro (erickttr@gmail.com) with modifications by:
% Vel (vel@latextemplates.com)
%
% Adapted from:
% J. Hrabe (http://www.magalien.com/public/nih_grants_in_latex.html)
%
% License:
% CC BY-NC-SA 3.0 (http://creativecommons.org/licenses/by-nc-sa/3.0/)
%
%%%%%%%%%%%%%%%%%%%%%%%%%%%%%%%%%%%%%%%%%

%----------------------------------------------------------------------------------------
%	PACKAGES AND OTHER DOCUMENT CONFIGURATIONS
%----------------------------------------------------------------------------------------

\documentclass[11pt,notitlepage]{article}

% Variabili per non ripetere i contatti mille volte, MODIFICARE QUI
\newcommand{\nomeStudente}{Giovanni Bruno}
\newcommand{\cognomeStudente}{Sanna}
\newcommand{\matricolaStudente}{1029744}
\newcommand{\emailStudente}{giovannibruno.sanna@studenti.unipd.com}
\newcommand{\telStudente}{+39 0000000000}

\newcommand{\nomeTutorAziendale}{Nome}
\newcommand{\cognomeTutorAziendale}{Cognome}
\newcommand{\emailTutorAziendale}{xxtutor@azienda.it}
\newcommand{\telTutorAziendale}{+39 0000000000}

\newcommand{\ragioneSocAzienda}{Soluzioni Software S.R.L.}
\newcommand{\indirizzoAzienda}{Via Via dei Ronchi, 21, 35127 Padova (PD)}
\newcommand{\sitoAzienda}{http://www.soluzioni-sw.it/}



% A note on fonts: As of 2013, NIH allows Georgia, Arial, Helvetica, and Palatino Linotype. LaTeX doesn't have Georgia or Arial built in; you can try to come up with your own solution if you wish to use those fonts. Here, Palatino & Helvetica are available, leave the font you want to use uncommented while commenting out the other one.
%\usepackage{palatino} % Palatino font
\usepackage[utf8x]{inputenc}%codifica
\usepackage{helvet} % Helvetica font
\usepackage{multirow}
\renewcommand*\familydefault{\sfdefault} % Use the sans serif version of the font
\usepackage[T1]{fontenc}
\linespread{1.2} % A little extra line spread is better for the Palatino font
\usepackage{fancyhdr} %pacchetto per le intestazioni
\usepackage{hyperref}
\usepackage{lipsum} % Used for inserting dummy 'Lorem ipsum' text into the template
\usepackage{amsfonts, amsmath, amsthm, amssymb} % For math fonts, symbols and environments
\usepackage{graphicx} % Required for including images
%\usepackage{booktabs} % Top and bottom rules for table
%\usepackage{wrapfig} % Allows in-line images
%\usepackage[labelfont=bf]{caption} % Make figure numbering in captions bold
\usepackage[top=0.5in,bottom=0.5in,left=0.5in,right=0.5in]{geometry} % Reduce the size of the margin
\pagestyle{empty}

\hyphenation{ionto-pho-re-tic iso-tro-pic fortran} % Specifies custom hyphenation points for words or words that shouldn't be hyphenated at all

\hypersetup{
	colorlinks=true,
	linkcolor=black,
	urlcolor=blue
}

%----------------------------------------------------------------------------------------
%	Creato da Mich - Updated by Simone Pessotto 04/08/2015
%----------------------------------------------------------------------------------------

\begin{document}
	
\noindent
\parbox{0.7\columnwidth}{Università degli Studi di Padova\\
	Piano di lavoro stage presso \ragioneSocAzienda{}\\
	\nomeStudente{} \cognomeStudente{} (\matricolaStudente{})}%matricola
\parbox{0.3\columnwidth}{
	\hfill \includegraphics[scale=0.08]{immagini/logo-unipd.png}}

\bigskip
\begin{center}
{\Huge \textbf{Analisi delle piattaforme di messaggistica}}\\ \bigskip
{\textbf{Milestone settimana \#1  \\   13/11/2017 - 17/11/2017}} \\ \bigskip
	%{\Large \textit{presso \ragioneSocAzienda{}}}\\ \bigskip
	{\Large \textit{\nomeStudente{} \cognomeStudente{}}}
\end{center}

%\section*{Contatti}
%\textbf{Studente:} \nomeStudente{} \cognomeStudente{}, \href{mailto:\emailStudente{}}{\emailStudente{}}, \telStudente{} \\
%\textbf{Tutor aziendale:} \nomeTutorAziendale{} \cognomeTutorAziendale{}, \href{mailto:\emailTutorAziendale{}}{\emailTutorAziendale{}}, \telTutorAziendale{} \\
%\textbf{Azienda:} \ragioneSocAzienda{}, \indirizzoAzienda{}, \href{\sitoAzienda{}}{\sitoAzienda{}}
\bigskip
\section*{Contenuto}
Tale documento contiene una breve relazione sulle piattaforme di messaggistica analizzate:
\begin{itemize}
		\item Telegram Messenger; 
		\item Skype Messenger; 
		\item Facebook Messenger;
		\item WeChat.
		
	\end{itemize} 
	Per ogni piattaforma analizzata viene riportata una breve descrizione, vantaggi e svantaggi, in termini di caratteristiche e funzionalità supportate. \\
	Inoltre viene individuata la piattaforma, tra quelle analizzate, scelta per lo sviluppo del bot.


\newpage



\noindent
\parbox{0.7\columnwidth}{Università degli Studi di Padova\\
	Piano di lavoro stage presso \ragioneSocAzienda{}\\
	\nomeStudente{} \cognomeStudente{} (\matricolaStudente{})}%matricola
\parbox{0.3\columnwidth}{
	\hfill \includegraphics[scale=0.08]{immagini/logo-unipd.png}}

\bigskip
\section*{Telegram Messenger}
\subsection*{Descrizione}
Telegram è un servizio di messaggistica istantanea basato su cloud ed erogato senza fini di lucro. I client ufficiali di Telegram sono open-source per diverse piattaforme. Caratteristiche di Telegram sono la possibilità di stabilire conversazioni cifrate ent-to-end, effettuare chiamate vocali cifrate end-to-end, scambiare messaggi vocali, videomessaggi, fotografie, video, stickers e file di qualsiasi tipo grandi fino a 1,5 GB.  \\
Oltre a inviare diversi tipi di file, Telegram è focalizzato sulla velocità. I messaggi si sincronizzano perfettamente in tutti i canali (e a tutti gli utenti) simultaneamente. \\
L'accessibilità è una caratteristica chiave di Telegram. Il suo sistema basato su cloud consente di accedere ai messaggi da qualsiasi dispositivo e da qualsiasi posizione, indipendentemente da dove siano stati originati. \\
La piattaforma è relativamente piccola, con circa 100 milioni di utenti, e non è dominante in molti paesi del mondo. Tuttavia, gli utenti che ha sono ben distribuiti e la sua base di utenti è in crescita.

\subsection*{Vantaggi}
\begin{itemize}
		\item Autorizzazione utente tramite chiave cifrata e numero di telefono;
		\item Supporta una crittografia a doppio strato. La crittografia server-client viene utilizzata nelle chat cloud, mentre le chat segrete utilizzano un ulteriore livello client-client. 
		\item Inline Bot: permette all'utente di inviare query al bot da una qualsiasi conversazione, senza dover sottoscrivere il bot e senza inviare nessun messaggio testuale;
		\item Possibilità di scegliere il data center più vicino a cui connettersi;
		\item Ottimizzazioni client-side: conferma di ricezione di un messaggio semplificata, ottimizzazione del download e upload da/sul server, raggruppamento degli update in assenza di connessione;
		\item Permette di scegliere la modalità di gestione degli update, se webhook o long-polling.
\end{itemize} 

\subsection*{Svantaggi}
\begin{itemize}
		\item Bacino contenuto di utenti;
		\item Documentazione ufficiale povera di contenuti e scarsamente approfondita; 
		\item Non integra NLP (Natural Language Processing) nativamente;
		\item Non fornisce nessun strumento integrato di analisi delle prestazioni del bot e delle interazioni con esso;
		\item Scarsa personalizzazione della UI nella conversazione.
\end{itemize} 

\bigskip
\section*{Skype Messenger}
\subsection*{Descrizione}
Chatbot Skype sono implementati attraverso l'utilizzo di Microsoft Bot Framework; esso fornisce strumenti e servizi per creare, distribuire e pubblicare bot, inclusi il servizio Bot di Azure, l'SDK del builder di Bot, Bot Framework Portal e il Bot Connector. \\
Il servizio Bot di Azure fornisce un ambiente integrato appositamente creato per lo sviluppo di bot. È possibile scrivere un bot, connettersi, testare, distribuire e gestirlo dal browser Web senza necessità di un editor separato o del controllo del codice sorgente; esso accelera il processo di sviluppo di un bot, offrendo cinque modelli di bot tra cui scegliere.  \\
Il Bot Framework include Bot Builder SDK, il quale permette di creare bot con C\# (Bot Builder SDK per .NET) o Javascript ( Bot Builder SDK per Nodejs). L'SDK contiene dialoghi integrati per gestire le interazioni dell'utente che vanno da un Sì / No di base a una disambiguazione complessa. I riconoscitori e i gestori di eventi integrati aiutano a guidare l'utente attraverso una conversazione. \\
Il Bot Framework Portal offre un posto comodo per registrare, connettere e gestire il bot. Fornisce inoltre strumenti diagnostici e un controllo della chat Web che è possibile utilizzare per incorporare il bot in una pagina Web. \\
Bot Framework supporta diversi canali popolari per la connessione di bot e persone. Gli utenti possono avviare conversazioni con il bot su qualsiasi canale con cui è stato configurato, inclusi email, Facebook, Skype, Slack e SMS. 

\subsection*{Vantaggi}
\begin{itemize}
		\item Disponibilità di SDK ufficiali in diversi linguaggi;
		\item Fornisce un'applicazione desktop per il test e debug del bot;  
		\item Integra API Microsoft per aggiungere potenti funzionalità e algoritmi di intelligenza artificiale. Tra queste, è molto interessante LUIS  (Language Understanding Intelligent Service), in grado di elaborare il linguaggio naturale utilizzando modelli linguistici predefiniti o personalizzati, oltre la capacità di rilevare sentimenti, frasi chiave, argomenti e lingua dal testo.
		\item Supporta diversi canali popolari e fornisce semplici connettori a tali canali. Questo permette agli utenti di avviare conversazioni con il bot su qualsiasi canale con cui è stato configurato, inclusi email, Facebook, Skype, Slack e SMS;
		\item Fornisce uno strumento integrato per il salvataggio e la gestione dello stato di una conversazione user-bot, fondamentale per conversazioni complesse.
		\item Fornisce uno strumento integrato per gestire il flusso della conversazione (Dialog);
		\item Documentazione ampia e dettagliata.
\end{itemize} 

\subsection*{Svantaggi}
\begin{itemize}
		\item Configurazione manuale dei sistemi per la protezione e la sicurezza delle comunicazioni;
		\item La possibilità di connettere il bot a diversi canali può rendere più lenta la velocità di risposta; 

\end{itemize} 

\bigskip
\section*{Facebook Messenger}
\subsection*{Descrizione}
Piattaforma di messaggistica istantanea sviluppata da Facebook, con più di un miliardo di utenti attivi. \\
Le caratteristiche principali sono la presenza di modelli personalizzabili che consentono di inviare contenuti più elaborati grazie ai messaggi strutturati. Questi modelli possono definire la gerarchia all'interno dei messaggi e creare interazioni complesse e dinamiche; essi incorporano alcuni elementi di interfaccia grafica (pulsanti, immagini, liste, ricevute, menù, ecc.). Inoltre prevede una visualizzazione web all'interno di una conversazione, la quale permette di visualizzare contenuti esterni basati su tecnologie web e con i quali si può interagire. \\
Una caratteristica peculiare è la presenza di strumenti per l'analisi delle prestazioni e raccoglimento di dati sulle interazioni degli utenti con il bot. Inoltre fornisce un protocollo di consegna, il quale permette a diversi bot di collaborare per una stessa pagina facebook, senza interferenze reciproche.

\subsection*{Vantaggi}
\begin{itemize}
		\item Integrazione di NLP (Natural Language Processing);
		\item Ricco insieme di API dedicate all'interfaccia utente; 
		\item Fornisce strumenti per la raccolta e analisi di diverse tipologie di dati (i.e. prestazioni, tipologie di interazioni e eventi, grado di utilizzo del bot, ecc.);
		\item Fornisce uno strumento per la visualizzazione di pagine web e interazione con esse;
		\item Fornisce menù persistenti nella conversazione, che permettono di visualizzare le possibili interazioni con il bot e guidano l'utente per passi;
		\item Prevede diversi modelli di messaggio personalizzabili (con pulsanti, liste, modelli generici, Open Graph, ricevute, ecc.);
		\item Documentazione ampia e dettagliata;
		\item Presenza numerosa di API.
\end{itemize} 

\subsection*{Svantaggi}
\begin{itemize}
		\item Sono necessarie procedure manuali per configurare la protezione delle conversazioni;
		\item Se il server non risponde ad una richiesta in pochi secondi, Facebook ritenta l'invio della richiesta al server; per elaborazioni che richiedono troppo tempo, il server riceve inutilmente la stessa richiesta più volte;
		\item Una volta creato il bot, esso deve essere sottoposto all'approvazione di Facebook.
		\item Esiste un limite per il numero di elementi e pulsanti che è possibile aggiungere ai modelli; c'è anche un limite per la lunghezza dei messaggi.
\end{itemize} 

\bigskip
\section*{WeChat}
\subsection*{Descrizione}
WeChat è un servizio di comunicazione disponibile per tutti i tipi di piattaforme; supporta chat di gruppo e messaggi vocali,foto,video e di testo,  ed è sviluppato dalla società cinese Tencent. La prima versione è stata distribuita a gennaio 2011. \\
Ad oggi conta circa 900 milioni di account, di cui più di un decimo in Cina; è stata fra le prime a fornire una piattaforma per lo sviluppo di chatbot, la quale ha contribuito alla sua diffusione in Cina insieme alla necessità degli utenti cinesi di poter fare acquisti online in maniera facile e sicura, soddisfatta dall'efficiente sistema di pagamento fornito dalla piattaforma WeChat. \\
Esistono due tipi di account chatbot ufficiali: account di abbonamento e account di servizio. Gli utenti principali degli account di abbonamento sono editori di contenuti. Gli abbonati di questi account ricevono automaticamente un elenco degli articoli pubblicati di recente. Il servizio clienti per i marchi e le aziende registrati su WeChat utilizza account di servizio per interagire con i loro clienti sull'applicazione.

\subsection*{Svantaggi}
\begin{itemize}
		\item Parte della documentazione è in lingua cinese;
		\item Le funzionalità del SDK per il pagamento non sono disponibili nella versione internazionale della piattaforma di sviluppo;
		\item Il governo cinese ha la possibilità di accedere ai dati sensibili degli utenti;
\end{itemize} 

\bigskip
\section*{Scelta conclusiva}
\end{document}