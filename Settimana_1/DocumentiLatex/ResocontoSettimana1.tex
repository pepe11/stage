%%%%%%%%%%%%%%%%%%%%%%%%%%%%%%%%%%%%%%%%%
% NIH Grant Proposal for the Specific Aims and Research Plan Sections
% LaTeX Template
% Version 1.0 (21/10/13)
%
% This template has been downloaded from:
% http://www.LaTeXTemplates.com
%
% Original author:
% Erick Tatro (erickttr@gmail.com) with modifications by:
% Vel (vel@latextemplates.com)
%
% Adapted from:
% J. Hrabe (http://www.magalien.com/public/nih_grants_in_latex.html)
%
% License:
% CC BY-NC-SA 3.0 (http://creativecommons.org/licenses/by-nc-sa/3.0/)
%
%%%%%%%%%%%%%%%%%%%%%%%%%%%%%%%%%%%%%%%%%

%----------------------------------------------------------------------------------------
%	PACKAGES AND OTHER DOCUMENT CONFIGURATIONS
%----------------------------------------------------------------------------------------

\documentclass[11pt,notitlepage]{article}

% Variabili per non ripetere i contatti mille volte, MODIFICARE QUI
\newcommand{\nomeStudente}{Giovanni Bruno}
\newcommand{\cognomeStudente}{Sanna}
\newcommand{\matricolaStudente}{1029744}
\newcommand{\emailStudente}{giovannibruno.sanna@studenti.unipd.com}
\newcommand{\telStudente}{+39 0000000000}

\newcommand{\nomeTutorAziendale}{Nome}
\newcommand{\cognomeTutorAziendale}{Cognome}
\newcommand{\emailTutorAziendale}{xxtutor@azienda.it}
\newcommand{\telTutorAziendale}{+39 0000000000}

\newcommand{\ragioneSocAzienda}{Soluzioni Software S.R.L.}
\newcommand{\indirizzoAzienda}{Via Via dei Ronchi, 21, 35127 Padova (PD)}
\newcommand{\sitoAzienda}{http://www.soluzioni-sw.it/}



% A note on fonts: As of 2013, NIH allows Georgia, Arial, Helvetica, and Palatino Linotype. LaTeX doesn't have Georgia or Arial built in; you can try to come up with your own solution if you wish to use those fonts. Here, Palatino & Helvetica are available, leave the font you want to use uncommented while commenting out the other one.
%\usepackage{palatino} % Palatino font
\usepackage[utf8x]{inputenc}%codifica
\usepackage{helvet} % Helvetica font
\usepackage{multirow}
\renewcommand*\familydefault{\sfdefault} % Use the sans serif version of the font
\usepackage[T1]{fontenc}
\linespread{1.2} % A little extra line spread is better for the Palatino font
\usepackage{fancyhdr} %pacchetto per le intestazioni
\usepackage{hyperref}
\usepackage{lipsum} % Used for inserting dummy 'Lorem ipsum' text into the template
\usepackage{amsfonts, amsmath, amsthm, amssymb} % For math fonts, symbols and environments
\usepackage{graphicx} % Required for including images
%\usepackage{booktabs} % Top and bottom rules for table
%\usepackage{wrapfig} % Allows in-line images
%\usepackage[labelfont=bf]{caption} % Make figure numbering in captions bold
\usepackage[top=0.5in,bottom=0.5in,left=0.5in,right=0.5in]{geometry} % Reduce the size of the margin
\pagestyle{empty}

\hyphenation{ionto-pho-re-tic iso-tro-pic fortran} % Specifies custom hyphenation points for words or words that shouldn't be hyphenated at all

\hypersetup{
	colorlinks=true,
	linkcolor=black,
	urlcolor=blue
}

%----------------------------------------------------------------------------------------
%	Creato da Mich - Updated by Simone Pessotto 04/08/2015
%----------------------------------------------------------------------------------------

\begin{document}
	
\noindent
\parbox{0.7\columnwidth}{Università degli Studi di Padova\\
	Piano di lavoro stage presso \ragioneSocAzienda{}\\
	\nomeStudente{} \cognomeStudente{} (\matricolaStudente{})}%matricola
\parbox{0.3\columnwidth}{
	\hfill \includegraphics[scale=0.08]{immagini/logo-unipd.png}}

\bigskip
\begin{center}
{\Huge \textbf{Resoconto Settimana 1}} \\ 
{\textbf{13/11/2017 - 17/11/2017}} \\ \bigskip
	{\Large \textit{presso \ragioneSocAzienda{}}}\\ \bigskip
	{\Large \textit{\nomeStudente{} \cognomeStudente{}}}
\end{center}

%\section*{Contatti}
%\textbf{Studente:} \nomeStudente{} \cognomeStudente{}, \href{mailto:\emailStudente{}}{\emailStudente{}}, \telStudente{} \\
%\textbf{Tutor aziendale:} \nomeTutorAziendale{} \cognomeTutorAziendale{}, \href{mailto:\emailTutorAziendale{}}{\emailTutorAziendale{}}, \telTutorAziendale{} \\
%\textbf{Azienda:} \ragioneSocAzienda{}, \indirizzoAzienda{}, \href{\sitoAzienda{}}{\sitoAzienda{}}

\section*{Scopo dello stage}

L’azienda si occupa di consulenza e sviluppo di software per la gestione aziendale, in particolare per la piccola
e media impresa. Fornisce servizi di consulenza e di riorganizzazione dei processi aziendali; inoltre si occupa
dello sviluppo di applicativi per la gestione aziendale, il CRM, la logistica e la Business Intelligence. \\ \\
Lo stage prevede l’analisi e lo sviluppo di un prototipo di ChatBot che fornisca informazioni su eventi legati alla
vita aziendale provenienti da CRM (Customer Relationship Management) o ERP (Enterprise Resource Planning). \\
Individuazione di una piattaforma di messaggistica che supporti i ChatBot e comparazione delle funzionalità
potenziali offerte; successiva realizzazione di un prototipo di ChatBot che notifichi, ad una lista di utenti definita,
eventi associati alla realtà aziendale derivanti dall’ERP o CRM. \\
Il progetto prevederà l’analisi, test e sviluppo di un prototipo di gestore di messaggi dall’azienda ai suoi dipendenti
utilizzando un ChatBot sviluppato su una delle piattaforme di messaggistica più note (WeChat, Telegram, Skype
Messenger). Lo studio individuerà una architettura che consentirà la distribuzione dei messaggi e lo sviluppo di
un prototipo che invii informazioni aziendali sia su richieste parametriche che su evento.

\bigskip
\section*{Contenuto del documento}
Nel presente documento è riportato: 
\begin{itemize}
		\item Milestone fissata per la settimana in oggetto;
		\item Riassunto del meeting settimanale avuto con il tutor aziendale;
		\item Descrizione delle attività eseguite durante la settimana;
		\item Raggiungimento risultati attesi.
	\end{itemize} 

\newpage



\noindent
\parbox{0.7\columnwidth}{Università degli Studi di Padova\\
	Piano di lavoro stage presso \ragioneSocAzienda{}\\
	\nomeStudente{} \cognomeStudente{} (\matricolaStudente{})}%matricola
\parbox{0.3\columnwidth}{
	\hfill \includegraphics[scale=0.08]{immagini/logo-unipd.png}}

\bigskip
\section*{Milestone}
Redazione di una breve relazione che descriva la piattaforma di messaggistica
scelta, sottolineando le motivazioni che hanno influenzato tale scelta e comparazione delle funzionalità potenziali offerte.

\bigskip
\section*{Riassunto meeting con tutor aziendale}
Di seguito sono riportati gli argomenti di discussione del meeting, avvenuto il 13/11/2017:
\begin{itemize}
		\item Sono state fissate le strategie per analisi della piattaforma messaggistica, in termini di caratteristiche rilevanti di queste ultime, funzionalità e strumenti messi a disposizione. In particolare, si sono ritenuti rilevanti: strumenti per garantire la sicurezza delle comunicazioni chatbot-server; disponibilità e funzionalità delle API; linguaggi di programmazione e framework disponibili per l'implementazione; strumenti supportati per l'analisi delle comunicazioni user-chatbot.
		\item Identificazione della tipologia di utenti che utilizzeranno il chatbot e delle informazioni più rilevanti a cui tali utenti vogliono accedere tramite comunicazione con il chatbot; 
		\item Illustrazione di uno schema rappresentante il meccanismo di funzionamento del chatbot, in termini di principali entità coinvolte e flusso dei dati.
\end{itemize} 




\bigskip
\section*{Attività}
Di seguito sono riportate le attività svolte durante la settimana 13/11/2017 - 17/11/2017:
\begin{itemize}
		\item Identificazione del target di utenti e delle informazioni a cui essi sono interessati. Tale attività è stata eseguita attraverso il supporto del tutor che ha fornito la tipologia di utenti che utilizzeranno il chatbot e tramite ricerca in rete delle informazioni di interesse per il target individuato;
		\item Analisi della piattaforme di messaggistica di Telegram, Skype Messenger, WeChat e Facebook Messenger; tale analisi è stata eseguita attraverso lettura della documentazione ufficiale, ricerca di articoli e blog correlati, lettura dei relativi forum e visione delle API disponibili;
		\item Redazione di una relazione in cui vengono descritte e comparate le piattaforme di messaggistica analizzate; individuazione della piattaforma scelta per lo sviluppo del chatbot. 
\end{itemize}


\newpage
\noindent
\parbox{0.7\columnwidth}{Università degli Studi di Padova\\
	Piano di lavoro stage presso \ragioneSocAzienda{}\\
	\nomeStudente{} \cognomeStudente{} (\matricolaStudente{})}%matricola
\parbox{0.3\columnwidth}{
	\hfill \includegraphics[scale=0.08]{immagini/logo-unipd.png}}

\bigskip
\section*{Raggiungimento risultati attesi}
Questa settimana si conferma che i risultati attesi sono stati raggiunti; ovvero è stata completata l'analisi delle piattaforme di messaggistica previste ed è stata redatta la relazione di tale analisi, contenente descrizione, vantaggi e svantaggi di ogni piattaforma analizzata e scelta conclusiva della piattaforma da utilizzare per l'implementazione del chatbot.

\end{document}